% Created 2024-04-13 Sat 19:19
% Intended LaTeX compiler: pdflatex
\documentclass[11pt]{article}
\usepackage[utf8]{inputenc}
\usepackage[T1]{fontenc}
\usepackage{graphicx}
\usepackage{longtable}
\usepackage{wrapfig}
\usepackage{rotating}
\usepackage[normalem]{ulem}
\usepackage{amsmath}
\usepackage{amssymb}
\usepackage{capt-of}
\usepackage{hyperref}
\author{Ringo Stark}
\date{\today}
\title{Ringo's blog}
\hypersetup{
 pdfauthor={Ringo Stark},
 pdftitle={Ringo's blog},
 pdfkeywords={},
 pdfsubject={},
 pdfcreator={Emacs 28.1 (Org mode 9.6)}, 
 pdflang={Vietnamese}}
\usepackage{biblatex}

\begin{document}

\maketitle
\tableofcontents


\section{Posts}
\label{sec:org6423547}
\subsection{Hello There!\hfill{}\textsc{some:tags}}
\label{sec:orgf094663}
General Kenobi
\subsection{ther post}
\label{sec:orgca7659c}
Keep bloggin

\subsection{Biến số của Intergrowth-21 trong SymjaBlog nghiêm túc của tôi}
\label{sec:org3449bcb}
Tôi quyết định sử dụng Intergrowth-21 làm chỉ số tham khảo chính, và lấy số liệu từ trang web chính thức ở đây \href{https://intergrowth21.tghn.org/fetal-growth/\#fg1}{Intergrowth-21 tables}. Từ các bảng số này, tham khảo cả bảng giá trị và z-score, tôi chỉ giữ lại 2 chỉ số cơ bản của phân phối là \texttt{mean} và \texttt{sd}. Sau đó tôi chuyển qua dạng Associations của Symja thì được các số liệu như sau.

\begin{verbatim}
{
bpdI21 = <| 14->{29.6,1.8}, 15->{32.6,1.8}, 16->{35.6,2}, 17->{38.8,2}, 18->{42,2.1}, 19->{45.2,2.2}, 20->{48.4,2.3}, 21->{51.7,2.4}, 22->{55,2.4}, 23->{58.2,2.6}, 24->{61.4,2.6}, 25->{64.5,2.8}, 26->{67.6,2.8}, 27->{70.6,2.9}, 28->{73.5,2.9}, 29->{76.3,3}, 30->{78.9,3.1}, 31->{81.4,3.1}, 32->{83.8,3.1}, 33->{85.9,3.3}, 34->{87.9,3.3}, 35->{89.7,3.4}, 36->{91.2,3.5}, 37->{92.5,3.6}, 38->{93.6,3.6}, 39->{94.4,3.8}, 40->{94.9,3.9} |>,

hcI21 = <|14->{97.9,5.6},15->{110.4,5.9},16->{122.9,6.3},17->{135.4,6.6},18->{147.9,6.9},19->{160.3,7.1},20->{172.5,7.4},21->{184.5,7.7},22->{196.3,7.9},23->{207.8,8.1},24->{219.1,8.3},25->{230,8.4},26->{240.5,8.6},27->{250.6,8.8},28->{260.4,8.9},29->{269.6,9.1},30->{278.4,9.2},31->{286.6,9.5},32->{294.4,9.6},33->{301.5,9.9},34->{308.1,10.1},35->{314.1,10.4},36->{319.4,10.8},37->{324.1,11.2},38->{328.1,11.6},39->{331.4,12.2},40->{333.9,13}|>,

flI21 = <| 14->{13.1,1.5}, 15->{16.3,1.6}, 16->{19.5,1.6}, 17->{22.5,1.7}, 18->{25.5,1.7}, 19->{28.5,1.7}, 20->{31.3,1.8}, 21->{34.1,1.8}, 22->{36.7,1.9}, 23->{39.4,1.9}, 24->{41.9,1.9}, 25->{44.4,1.9}, 26->{46.7,2.1}, 27->{49.1,2}, 28->{51.3,2.1}, 29->{53.4,2.2}, 30->{55.5,2.3}, 31->{57.5,2.3}, 32->{59.5,2.3}, 33->{61.3,2.5}, 34->{63.1,2.5}, 35->{64.8,2.6}, 36->{66.4,2.7}, 37->{68,2.8}, 38->{69.4,3}, 39->{70.8,3.1}, 40->{72.1,3.3} |>,

acI21 = <| 14->{80.6,4.1}, 15->{91.9,4.8}, 16->{103.2,5.4}, 17->{114.4,6}, 18->{125.6,6.6}, 19->{136.7,7.1}, 20->{147.7,7.7}, 21->{158.7,8.1}, 22->{169.6,8.6}, 23->{180.4,9.1}, 24->{191.2,9.4}, 25->{201.8,10}, 26->{212.4,10.4}, 27->{222.9,10.8}, 28->{233.3,11.3}, 29->{243.6,11.8}, 30->{253.8,12.4}, 31->{263.9,13}, 32->{273.9,13.6}, 33->{283.8,14.4}, 34->{293.6,15.1}, 35->{303.3,16}, 36->{312.8,17.1}, 37->{322.3,18.1}, 38->{331.6,19.4}, 39->{340.8,20.8}, 40->{349.8,22.4} |>,

weightI21 = <| 14->{ 0.01, 0.01}, 15->{ 0.01, 0.01}, 16->{ 0.01, 0.01}, 17->{ 0.01, 0.01}, 18->{ 0.01, 0.01}, 19->{ 0.01, 0.01}, 20->{ 0.01, 0.01}, 21->{ 0.01, 0.01}, 22->{525,40}, 23->{592,50}, 24->{668,63}, 25->{756,77}, 26->{856,94}, 27->{969,114}, 28->{1097,137}, 29->{1239,163}, 30->{1396,192}, 31->{1568,224}, 32->{1755,256}, 33->{1954,290}, 34->{2162,323}, 35->{2378,353}, 36->{2594,380}, 37->{2806,401}, 38->{3006,415}, 39->{3186,420}, 40->{3338,416} |>
}
\end{verbatim}

Lưu ý là trong Intergrowth-21, các tuổi thai từ 14 đến 21 tuần không có phân bố cân nặng thai, nên nếu các tính toán ở tuổi thai này sẽ sinh ra lỗi do không tồn tại giá trị. Vì vậy tôi đồng nhất hóa các biến số ở đây bằng 0.01 (không thể bằng 0 được, vì khi đó cũng sinh ra lỗi) để các tính toán được thông suốt, nhưng hoàn toàn không có ý nghĩa nữa.

\subsection{Hàm Hadlock có so sánh tuổi thai theo tiêu chuẩn Intergrowth-21}
\label{sec:org42f3810}
Việc tính toán cân nặng thai sẽ không có ý nghĩa, nếu không biết cân nặng đó có đúng chuẩn so với tuổi thai hiện tại hay không. Vì vậy, nhất thiết phải có một hàm so sánh cân nặng thai hiện tại và suy ra bách phân vị mà thai đang đứng.

Tiêu chuẩn số liệu nên dựa vào chuẩn nào, Intergrowth-21 chăng?

EDITED: \textit{[2024-02-26 Mon 21:12]}
Đây là đoạn mã hoàn chỉnh của hàm Hadlock mở rộng có tính percentiles theo tiêu chuẩn Intergrowth-21 cho thai

\begin{verbatim}
{
{
m = 1,

s = 2,

bpdI21 = <| 14->{29.6,1.8}, 15->{32.6,1.8}, 16->{35.6,2}, 17->{38.8,2}, 18->{42,2.1}, 19->{45.2,2.2}, 20->{48.4,2.3}, 21->{51.7,2.4}, 22->{55,2.4}, 23->{58.2,2.6}, 24->{61.4,2.6}, 25->{64.5,2.8}, 26->{67.6,2.8}, 27->{70.6,2.9}, 28->{73.5,2.9}, 29->{76.3,3}, 30->{78.9,3.1}, 31->{81.4,3.1}, 32->{83.8,3.1}, 33->{85.9,3.3}, 34->{87.9,3.3}, 35->{89.7,3.4}, 36->{91.2,3.5}, 37->{92.5,3.6}, 38->{93.6,3.6}, 39->{94.4,3.8}, 40->{94.9,3.9} |>,

hcI21 = <|14->{97.9,5.6},15->{110.4,5.9},16->{122.9,6.3},17->{135.4,6.6},18->{147.9,6.9},19->{160.3,7.1},20->{172.5,7.4},21->{184.5,7.7},22->{196.3,7.9},23->{207.8,8.1},24->{219.1,8.3},25->{230,8.4},26->{240.5,8.6},27->{250.6,8.8},28->{260.4,8.9},29->{269.6,9.1},30->{278.4,9.2},31->{286.6,9.5},32->{294.4,9.6},33->{301.5,9.9},34->{308.1,10.1},35->{314.1,10.4},36->{319.4,10.8},37->{324.1,11.2},38->{328.1,11.6},39->{331.4,12.2},40->{333.9,13}|>,

flI21 = <| 14->{13.1,1.5}, 15->{16.3,1.6}, 16->{19.5,1.6}, 17->{22.5,1.7}, 18->{25.5,1.7}, 19->{28.5,1.7}, 20->{31.3,1.8}, 21->{34.1,1.8}, 22->{36.7,1.9}, 23->{39.4,1.9}, 24->{41.9,1.9}, 25->{44.4,1.9}, 26->{46.7,2.1}, 27->{49.1,2}, 28->{51.3,2.1}, 29->{53.4,2.2}, 30->{55.5,2.3}, 31->{57.5,2.3}, 32->{59.5,2.3}, 33->{61.3,2.5}, 34->{63.1,2.5}, 35->{64.8,2.6}, 36->{66.4,2.7}, 37->{68,2.8}, 38->{69.4,3}, 39->{70.8,3.1}, 40->{72.1,3.3} |>,

acI21 = <| 14->{80.6,4.1}, 15->{91.9,4.8}, 16->{103.2,5.4}, 17->{114.4,6}, 18->{125.6,6.6}, 19->{136.7,7.1}, 20->{147.7,7.7}, 21->{158.7,8.1}, 22->{169.6,8.6}, 23->{180.4,9.1}, 24->{191.2,9.4}, 25->{201.8,10}, 26->{212.4,10.4}, 27->{222.9,10.8}, 28->{233.3,11.3}, 29->{243.6,11.8}, 30->{253.8,12.4}, 31->{263.9,13}, 32->{273.9,13.6}, 33->{283.8,14.4}, 34->{293.6,15.1}, 35->{303.3,16}, 36->{312.8,17.1}, 37->{322.3,18.1}, 38->{331.6,19.4}, 39->{340.8,20.8}, 40->{349.8,22.4} |>,

weightI21 = <| 14->{ 0.01, 0.01}, 15->{ 0.01, 0.01}, 16->{ 0.01, 0.01}, 17->{ 0.01, 0.01}, 18->{ 0.01, 0.01}, 19->{ 0.01, 0.01}, 20->{ 0.01, 0.01}, 21->{ 0.01, 0.01}, 22->{525,40}, 23->{592,50}, 24->{668,63}, 25->{756,77}, 26->{856,94}, 27->{969,114}, 28->{1097,137}, 29->{1239,163}, 30->{1396,192}, 31->{1568,224}, 32->{1755,256}, 33->{1954,290}, 34->{2162,323}, 35->{2378,353}, 36->{2594,380}, 37->{2806,401}, 38->{3006,415}, 39->{3186,420}, 40->{3338,416} |>,

ageI21 = Round(age),

BPD1 = bpd / 10,

HC1 = hc / 10,

FL1 = fl / 10,

AC1 = ac / 10,

WEIGHT = Round(10^
If (Or (AC1 <= 0, FL1 <= 0),
    -1,
    If (HC1 <= 0,
        If (BPD1 <= 0,
            1.304 + 0.05281 * AC1 + 0.1938 * FL1 - 0.004 * AC1 * FL1,
            1.335 - 0.0034 * AC1 * FL1 + 0.0316 * BPD1 + 0.0457 * AC1 + 0.1623 * FL1),
        If (BPD1 <= 0,
            1.326 - 0.00326 * AC1 * FL1 + 0.0107 * HC1 + 0.0438 * AC1 + 0.158 * FL1,
            1.3596 - 0.00386 * AC1 * FL1 + 0.0064 * HC1 + 0.00061 * BPD1 * AC1 + 0.0424 * AC1 + 0.174 * FL1)
    )
)
),

bpdPercentile = Round(100 * CDF(NormalDistribution(bpdI21[ageI21][[m]],bpdI21[ageI21][[s]]),bpd)),

hcPercentile = Round(100 * CDF(NormalDistribution(hcI21[ageI21][[m]],hcI21[ageI21][[s]]),hc)),

flPercentile = Round(100 * CDF(NormalDistribution(flI21[ageI21][[m]],flI21[ageI21][[s]]),fl)),

acPercentile = Round(100 * CDF(NormalDistribution(acI21[ageI21][[m]],acI21[ageI21][[s]]),ac)),

weightPercentile = Round(100 * CDF(NormalDistribution(weightI21[ageI21][[m]],weightI21[ageI21][[s]]),WEIGHT))
}[[14]],
  hcPercentile, flPercentile, acPercentile, WEIGHT, weightPercentile}
\end{verbatim}

NOTE: Ở đây tôi sử dụng một thủ thuật để giải quyết vấn đề chuỗi lệnh của symja chưa được hoàn thiện. Chuỗi lệnh của symja được nối tiếp nhau bằng dấu chấm phẩy ";", và câu lệnh cuối cùng sẽ được in ra màn hình, điều đó ẩn ý là câu lệnh cuối là câu lệnh in ra các biến số cần in. Nhưng trong phiên bản trên điện thoại, nó lại in ra câu lệnh đầu tiên, tôi thử chèn câu lệnh in lên đầu, nhưng không đạt được kết quả mong muốn. Vì vậy, tôi sử dụng một thủ thuật trong List, nghĩa là toàn bộ chương trình này chỉ là một chuỗi, chuỗi đầu tiên tính toán tất cả các công thức cần thiết và in ra biến số đầu tiên, các biến số sau đó được tùy chọn để in ra tiếp theo, nhờ đã có kết quả tính toán ở chuỗi trước.
\end{document}